\documentclass[a4paper,10pt]{jarticle}

\usepackage{graphicx, amssymb, amsmath, color, url}

\newtheorem{thm}{定理}[section]
\newtheorem{prop}[thm]{命題}
\newtheorem{lemma}[thm]{補題}
\newtheorem{cor}[thm]{系}
\newtheorem{defi}[thm]{定義}
\newtheorem{rem}[thm]{注意}
\newtheorem{ex}[thm]{例}


\pagestyle{plain}

\title{NonParametrics 2024}
\author{Yasuyuki Matsumura\thanks{M1 Student at Graduate School of Economics, Kyoto University. yasu0704xx \text{@} gmail.com.}}
\date{\today} %今日の日付が自動的に入る

%%%%%%    TEXT START    %%%%%%
\begin{document}
\maketitle % 文書のタイトルを生成

\setcounter{section}{-1}
\section{はじめに}

gitリポジトリ「NonParametrics2024」は,
京都大学大学院経済学研究科で2024年度に開講されている
「計量経済学1」と「計量経済学2」(西山慶彦先生ご担当)の演習において,
松村が発表の際に用いた資料を公開するために作成したものです.
アップロードされている資料の二次利用はご遠慮ください.


\section{参考文献:英語のテキスト}

\begin{itemize}
  \item Hansen, B. E. (2022)
        \textit{Econometrics,} Princeton University Press.
        \begin{itemize}
          \item 計量経済学の超ド定番の教科書なので,詳細は省略.
        \end{itemize}
  
  \item Li, Q. and J. S. Racine. (2007)
        \textit{Nonparametric Econometrics: Theory and Practice,} 
        Princeton University Press.

        \begin{itemize}
        \item 京大経研(2024年)「計量経済学1,2」(西山慶彦先生ご担当)で輪読している教科書.
        \item ノンパラの教科書の定番らしい.
        \item 過去には,Hansen先生(2009年,University of Wisconsin)や末石先生(2014年,京大)のトピックコースでも使用していたらしい.
        \item ECON 718 NonParametric Econometrics Spring 2009 Bruce Hansen 
        \item \url{https://users.ssc.wisc.edu/~bhansen/718/718.htm}
        \item セミ・ノンパラメトリック計量分析 
        \item \url{https://sites.google.com/site/naoyasueishij/teaching/nonpara?authuser=0}
        \end{itemize}
  
  \item van der Vaart, A. W. (2000) 
  \textit{Asymptotic Statistics,} Cambridge University Press.
        \begin{itemize}
          \item 数理統計学の超ド定番の教科書なので,詳細は省略.
          \item Chapters 24, 25がノンパラ,セミパラを扱っている.
        \end{itemize}

\end{itemize}


\section{参考文献:日本語のテキスト}

\begin{itemize}
  \item 久保木久孝,鈴木武 (2015) 『セミパラメトリック推測と経験過程』朝倉書店.
        \begin{itemize}
          \item 最近買ったところだから何とも言えない.これから読む.
          \item セミパラというよりEmpirical Processの勉強に使う本っぽい(それが目的で買った).
        \end{itemize}  
  
  \item 清水泰隆 (2021) 『統計学への確率論,その先へ:ゼロからの測度論的理解と漸近理論への架け橋』内田老鶴圃.
        \begin{itemize}
          \item 測度論をひととおり勉強できる.優収束定理等の積分と極限の扱いを勉強するのに役立った.
        \end{itemize} 
    
  \item 清水泰隆 (2023) 『統計学への漸近論,その先は:現代の統計リテラシーから確率過程の統計学へ』内田老鶴圃.
        \begin{itemize}
          \item コアノメの副読本みたいな感じで読んでる.ノンパラは5章.
        \end{itemize}

  \item 末石直也 (2015) 『計量経済学:ミクロデータ分析へのいざない』日本評論社.
        \begin{itemize}
          \item ノンパラを扱ってるのは9章.
          \item パラメトリックの枠は出ないけど,分位点回帰,打ち切りモデル,Binary Choiceモデルなどなど,
                ノンパラ・セミパラで推定したいモデルの基礎がひととおり説明されている.
        \end{itemize}
  
  \item 末石直也 (2024) 『データ駆動型回帰分析:計量経済学と機械学習の融合』日本評論社.
        \begin{itemize}
          \item ノンパラ:3章,セミパラ:4章.
          \item お気持ち部分を丁寧に概観できる.
        \end{itemize}

  \item 西山慶彦,人見光太郎 (2023) 『ノン・セミパラメトリック統計解析(理論統計学教程:数理統計の枠組み)』共立出版.
        \begin{itemize}
          \item だいたい全部ここに載っている.
          \item ややこしすぎる証明は元ペーパーを参照する形でカットされていて,読み進めやすい気がする.
          \item Li and Racine (2007) の輪読会の準備をするときは,これで予習してます.
        \end{itemize}
\end{itemize}

\section{参考文献:Paper}
いっぱいあるから省略.




























\end{document}
