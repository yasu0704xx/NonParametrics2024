\documentclass[xcolor=svgnames,dvipdfmx,cjk]{beamer} 
\AtBeginDvi{\special{pdf:tounicode 90ms-RKSJ-UCS2}} 
\usetheme{Madrid}
\setbeamercolor{background canvas}{bg=Snow}
\usecolortheme[named=RoyalBlue]{structure}
\usefonttheme{professionalfonts}
\setbeamertemplate{theorems}[numbered]
\newtheorem{thm}{Theorem}[section]
\newtheorem{proposition}[thm]{Proposition}
\theoremstyle{example}
\newtheorem{exam}[thm]{Example}
\newtheorem{remark}[thm]{Remark}
\newtheorem{question}[thm]{Question}
\newtheorem{prob}[thm]{Problem}
\usepackage{bbm}
\usepackage{ascmac}

\begin{document} 

%%%%%講演に関する情報%%%%%%%%%%%%%%%%%%%%%%%%%%%%%%%%%%%%%%%%
\title[Li and Racine (2007, Chapter 8)]{Semiparametric Single Index Models} 
\author[Y. Matsumura]{Yasuyuki Matsumura}          
\institute[]{Graduate School of Economics, Kyoto University} 
\date{\today}


%%%%%タイトルページ%%%%%%%%%%%%%%%%%%%%%%%%%%%%%%%%%%%%%%%%%%
\begin{frame}                  
\titlepage                     
\end{frame}


%%%%%目次のページ%%%%%%%%%%%%%%%%%%%%%%%%%%%%%%%%%%%%%%%%%%%%
\begin{frame}                  
\tableofcontents
\end{frame}
%本文中に挿入するSECTION環境によって定義された目次が自動で反映される


%%%%%%%%%%%%%%%%%%%%%%%%%%%%%%%%%%%%%%%%%%%%%%%%%%%%%%%%%%%%%%%
%%%%%%%%%%%%%%%%%%%%%%%%%%%%%%%%%%%%%%%%%%%%%%%%%%%%%%%%%%%%%%%
%%%%%%%%%%%%%%%%%%%%%%%%%%%%%%%%%%%%%%%%%%%%%%%%%%%%%%%%%%%%%%%

\section{Identification Conditions}
\begin{frame}
  \tableofcontents[currentsection]
\end{frame}








\section{Estimation: Ichimura (1993)}
\begin{frame}
  \tableofcontents[currentsection]
\end{frame}









\section{Direct Semiparametric Estimators for $\beta$}
\begin{frame}
  \tableofcontents[currentsection]
\end{frame}









\section{Bandwidth Selection}
\begin{frame}
  \tableofcontents[currentsection]
\end{frame}









\section{Klein and Spady (1993)}
\begin{frame}
  \tableofcontents[currentsection]
\end{frame}









\section{Lewbel (2000)}
\begin{frame}
  \tableofcontents[currentsection]
\end{frame}









\section{Manski's (1975) Maximum Score Estimator}
\begin{frame}
  \tableofcontents[currentsection]
\end{frame}









\section{Horowitz's (1992) Smoothed Maximum Score Estimator}
\begin{frame}
  \tableofcontents[currentsection]
\end{frame}









\section{Han's (1987) Maximum Rank Estimator}
\begin{frame}
  \tableofcontents[currentsection]
\end{frame}









\section{Multinomial Discrete Choice Models}\begin{frame}
  \tableofcontents[currentsection]
\end{frame}









\section{Ai's (1997) Semiparametric Maximum Likelihood Approach}
\begin{frame}
  \tableofcontents[currentsection]
\end{frame}









%%%%%%%%%%%%%%%%%%%%%%%%%%%%%%%%%%%%%%%%%%%%%%%%%%%%%%%%%%%%%%%
%%%%%%%%%%%%%%%%%%%%%%%%%%%%%%%%%%%%%%%%%%%%%%%%%%%%%%%%%%%%%%%
%%%%%%%%%%%%%%%%%%%%%%%%%%%%%%%%%%%%%%%%%%%%%%%%%%%%%%%%%%%%%%%

\section{References}
\begin{frame}
  \tableofcontents[currentsection]
\end{frame}

\begin{frame}{References}
  \begin{itemize}
    \item Li, Q. and J. S. Racine, (2007). 
          \textit{Nonparametric Econometrics: Theory and Practice,} 
          Princeton University Press.
    \item 末石直也 (2024) 『データ駆動型回帰分析:計量経済学と機械学習の融合』日本評論社.
    \item 西山慶彦,人見光太郎 (2023) 『ノン・セミパラメトリック統計解析(理論統計学教程:数理統計の枠組み)』共立出版.
    \item このほかに,ECON 718 NonParametric Econometrics (Bruce Hansen, Spring 2009, University of Wisconsin-Madison) や,
          セミノンパラメトリック計量分析(末石直也,2014年度後期,京都大学大学院経済学研究科)のレクチャーノートを参照した. 
  \end{itemize}
\end{frame}

\end{document}